% LaTeX resume using res.cls
\documentclass[margin]{res}
%\usepackage{helvetica} % uses helvetica postscript font (download helvetica.sty)
%\usepackage{newcent}   % uses new century schoolbook postscript font 
\setlength{\textwidth}{5.1in} % set width of text portion

\begin{document}

% Center the name over the entire width of resume:
 \moveleft.5\hoffset\centerline{\large\bf Pham Trong Truong}
% Draw a horizontal line the whole width of resume:
 \moveleft\hoffset\vbox{\hrule width\resumewidth height 1pt}\smallskip
% address begins here
% Again, the address lines must be centered over entire width of resume:
 \moveleft.5\hoffset\centerline{Birthday: 02/09/1986}
 \moveleft.5\hoffset\centerline{Mobile: +81 90 6022 0187}
 \moveleft.5\hoffset\centerline{Email: truongptk30a3@gmail.com}

\begin{resume}
 
\section{OBJECTIVE}  Embedded Software Specialist. 
 
\section{EDUCATION} {\sl Hanoi University of Science and Technology} \hfill 09/2004 - 06/2009
                 \begin{itemize}  \itemsep -2pt  %reduce space between items
                 \item Center for Talents Training.
                 \item Major: Automatic Control/Control Engineering and Automatics.
                 \item GPA: 8.15/10.
                 \end{itemize}

\section{TECHNICAL \\ SKILLS}
        {\sl Programming Language:} C/C++. \\
        {\sl Video Coding Format \& Media File Format:} MPEG2-Video/H.262, AVC/H.264, HEVC/H.265, MPEG2-TS, MP4, MKV.\\
        {\sl OSS:} FFmpeg, GStreamer, Linux Kernel, HEVC Model(HM), AVC Joint Model(JM).\\ 
        {\sl OS \& Tool:} Windows, GNU/Linux, Emacs, Vim, Altium (PCB design).\\
        {\sl Other:} Microcontroller (8051, PIC, AVR), Raspberry Pi, OpenMAX IL.

\section{FOREIGN \\ LANGUAGE}
         {\sl Vietnamese:} Native  \\
         {\sl English:}  Intermediate\\
         {\sl Japanese:} \~\ N2  \\

 
\section{EMPLOYMENT \\ HISTORY}
		{\sl VTI Co Limited (http://vti.com.vn)}  \hfill 08/2018 - Present
                 \begin{itemize}  \itemsep -2pt  %reduce space between items
                 \item Job Title: Software Engineer.
		 \item Job Description:  Design, Coding. 
                 \end{itemize}

		{\sl FPT Japan (https://www.fpt-software.jp/fpt-japan)}  \hfill 07/2013 - 07/2018
                 \begin{itemize}  \itemsep -2pt  %reduce space between items
                 \item Job Title: Software Engineer, Bridge SE.
		 \item Job Description:  Design, Coding, Offshore Management. 
                 \end{itemize}

		 {\sl FPT Software Ha Noi (https://www.fpt-software.com/hanoi)}  \hfill 07/2009 - 07/2013
                 \begin{itemize}  \itemsep -2pt  %reduce space between items
                 \item Job Title: Software Engineer.
		 \item Job Description: Design, Coding.
                 \end{itemize}

		 {\sl Binh Anh Electronics (http://binhanh.vn)}   \hfill 11/2007 - 02/2009
                 \begin{itemize}  \itemsep -2pt  %reduce space between items
		 \item Job Title: Embedded Software Developer. 
		 \item Job Description: Design, Coding. (Part-time job)
                 \end{itemize}
        
\section{EXPERIENCE}
                {\sl 3D TOF sensor.} \hfill            8/2018 - Present \\
		 Development framework and driver of 3D TOF sensor on Android.\\
		 - Responsibility: In charge a part of framework.


                {\sl Linux Kernel BSP.} \hfill            5/2018 - 7/2018 \\
		 Investigate issue when upgrading Linux kernel 3.18.24 to 3.18.82 on SOC of customer.\\
		 - Responsibility: Design, Analysis USB module.

                {\sl eT-Kenel BSP.} \hfill            10/2017 - 04/2018 \\
		 Porting (design, coding, testing) HD-DMAC, HS-SPI, I2S, ENC/DEC driver on eT-Kernel to new SOC of customer.\\
		 - Responsibility: BrSE (Communicate between offshore team and customer).


                {\sl CAN-USB tool.} \hfill            04/2017 - 10/2017 \\
		 Develop tool which instead of ECUs in CAN network to test Automotive Head Unit.\\
		 - Responsibility: BrSE (Communicate between offshore team and customer).

                {\sl 4K Camera Recoder.} \hfill            08/2015 - 04/2017 \\
		 Develop core video encoder of security \& professional camera, supports MPEG2-Video/H.262, AVC/H.264, HEVC/H.265.\\
		 - Responsibility: Video Encoder (Coding, Testing, Bugfixing).
   

                {\sl 4K Digital Television.} \hfill            07/2013 - 07/2015 \\
		 Develop firmware of 4K digital television, supports MPEG2-TS, MP4 file format.\\
		 - Responsibility: Test Program, Demuxer.

                {\sl TIVI-Driver} \hfill            02/2013 - 06/2013 \\
		 Testing linux device driver (GPIO, Timer, SPI, I2C).\\
		 - Responsibility: Create test program..

                {\sl G-BOOK} \hfill            09/2012 - 02/2013 \\
		 Develop telematics subscription service in car system.\\
		 - Responsibility: Develop feature display received message from call center.


                {\sl Android4.0 (ICS) BSP BugFixing} \hfill            04/2012 - 09/2012\\
		 Resolve problem of linux driver, media framework when upgrading from Android 2.3 to 4.0.\\
		 - Responsibility: Testing SPI, I2C, GPIO driver, BugFix driver \& media framework.


                {\sl Audio Video Decoder} \hfill            08/2009 - 04/2012 \\
                 Develop firmware of digital television (Digital Video Broadcasting - Terrestrial).\\
                 Support trick play (slow, fast) with MP4, MKV, ASF, AVI file format.\\
                 Design firmware by API of OpenMax-IL and integrate to Stagefright (Android Media Framework).\\
		 - Responsibility: AVC/H.264 decoder. 


                {\sl Utility Management System} (Binh Anh Electronics). \hfill        11/2007-02/2009 \\
                Using PIC16F877A, RF module, LED 7 Segment to develop some utility solution.\\
		- Responsibility: Design, Implement firmware \& PCB.

 
\section{OTHER \\ ACTIVITY}  
		{\sl Blog (Vietnamese only):} https://vcostudy.com \\
		{\sl Side project:} https://vcostudy.com/spj

\section{HOBBY}  History, Physics, Beer.


\end{resume}
\end{document}




