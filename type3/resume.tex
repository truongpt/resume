\documentclass{resume}
\usepackage{latexsym}
\usepackage[empty]{fullpage}
\usepackage{titlesec}
\usepackage{marvosym}
\usepackage[usenames,dvipsnames]{color}
\usepackage{verbatim}
\usepackage{enumitem}
\usepackage[hidelinks]{hyperref}
\usepackage{fancyhdr}
\usepackage[english]{babel}
\usepackage[left=0.75in,top=0.6in,right=0.75in,bottom=0.6in]{geometry}

\begin{document}

  \begin{tabular*}{\textwidth}{l@{\extracolsep{\fill}}r}
   \textbf{\href{https://www.linkedin.com/in/truongpt/}{\Large Truong Pham}} & Email : \href{mailto:truongptk30a3@gmail.com}{truongptk30a3@gmail.com}\\
   \href{https://gravieb.wordpress.com/}{https://gravieb.wordpress.com} & Mobile : +81-906-022-0187 \\
  \end{tabular*}

  \begin{rSection}{Objective}
    Specialist embedded system.
  \end{rSection}

  \begin{rSection}{Education}
    {\bf Phan Boi Chau Specialized High School, Nghe An - Viet Nam} \hfill {\em 09/2001 - 06/2004} \\ 
    { 2rd prize Vietnam National Physics Contest for high school - 2004 } \\
    { Bronze medal 5th Asian Physics Olympiad - 2004 }

    {\bf Hanoi University of Science and Technology} \hfill {\em 09/2004 - 06/2009} \\ 
    { B.S. in Automatic Control/Electrical Engineering department} \\
    { Graduation thesis: 48V DC - 220V AC, 500W Converter} \\ 
    { 3rd prize research student of Electrical Engineering department - 2009} \\
    { Overall GPA: 8.15/10 }
  \end{rSection}

  \begin{rSection}{Technical Strengths}
    Programming Language: C/C++. \\
    Multimedia processing: MPEG2-Video/H.262, AVC/H.264, HEVC/H.265, MPEG2-TS. \\
    OSS: FFmpeg, GStreamer, Linux Kernel, HEVC Model(HM), AVC Joint Model(JM). \\
    Hardware: Microcontroller 8/16/32bit, SOC base on ARM, MISP.
  \end{rSection}

  \begin{rSection}{Foreign Language}
    Vietnamese: Native \\
    English: Intermediate, Communicate fluently \\
    Japanese: nearly N2
  \end{rSection}

  \begin{rSection}{Employment \& Experience}
  
    \begin{rSubsection}{\href{https://www.vti.com.vn/company-vti-japan/}{\underline{VTI Co Limited}}}{08/2018 - Present}{Embedded Sofware Engineer}{Tokyo, Japan}
    \item \textbf{3D TOF sensor}: Develop framework and driver of 3D TOF sensor on Android. (08/2018 - Present)
    \newline \textit{Responsibility}: Implement image filter using arm NEON instruction. Investigate improve performance.
    \end{rSubsection}
  
    \begin{rSubsection}{\href{https://www.fpt-software.jp/company-information/fpt-japan/}{\underline{FPT Japan}}}{07/2013 - 07/2018}{Embedded Software Engineer, Bridge System Engineer}{Tokyo, Japan}
    \item \textbf{Linux Kernel BSP}: Investigate to port the property device driver part on SOC of the customer to upgrade Linux kernel 3.18.24 to 3.18.82. (5/2018 - 7/2018)
    \newline \textit{Responsibility}: Investigate USB module.

    \item \textbf{eT-Kenel BSP}: Porting HD-DMAC, HS-SPI, I2S, ENC/DEC driver on eTKernel to new SOC. (10/2017 - 04/2018)
    \newline \textit{Responsibility}: Offshore manager. General support technical \& management.

    \item \textbf{CAN-USB tool}: Develop tool which instead of ECUs in CAN network to test Automotive Head Unit. (04/2017 - 10/2017) 
    \newline \textit{Responsibility}: Offshore manager. General support technical \& management.

    \item \textbf{4K Camera Recorder}: Develop firmware of security \& 4K professional camera, the most popular 4K recorder camera product in Japan. (08/2015 - 04/2017)
    \newline \textit{Responsibility}: Member of video encoder team, support H.264, H.265.

    \item \textbf{4K Digital Television}: Develop firmware of 4K digital television, the most popular 4K television product in Japan. (07/2013 - 07/2015)
    \newline \textit{Responsibility}: Member of demuxer team, handle input multimedia data, support MPEG2-TS, MP4 file format.

    \end{rSubsection}

    \begin{rSubsection}{\href{https://www.fpt-software.com/}{\underline{FPT Software}}}{07/2009 - 07/2013}{Embedded Software Developer, Team leader}{Hanoi, Vietnam}
    \item \textbf{TIVI-Driver}: Evaluate GPIO, Timer, SPI, I2C driver on linux kernel. (02/2013 - 06/2013)
    \newline \textit{Responsibility}: Develop test program, test \& investigate problem.

    \item \textbf{G-BOOK}: Develop telematics subscription service in-car system. (09/2012 - 02/2013)
    \newline \textit{Responsibility}: Develop feature display received message from call center.

    \item \textbf{Android4.0 BSP BugFixing}: Resolve problem of linux driver, media framework when upgrading from Android 2.3 to 4.0. (04/2012 - 09/2012)
    \newline \textit{Responsibility}: Investigate and support problem of SPI, I2C, GPIO driver and media framework. 

    \item \textbf{Audio Video Decoder 2}: Develop core multimedia processing firmware following to API of OpenMax-IL to integrate with Stagefright on Android Media Framework. (11/2011 - 04/2012)
    \newline \textit{Responsibility}: Develop video decoder component, support H.264 video coding format.

    \item \textbf{Audio Video Decoder 1}: Develop firmware of digital television support DVB-T. Support trick play (slow, fast) with MP4, MKV, ASF, AVI file format. (08/2009 - 11/2011)
    \newline \textit{Responsibility}: AVC/H.264 decoder, Video display.
    \end{rSubsection}

    \begin{rSubsection}{\href{https://bagps.vn/}{\underline{Binh Anh Electronics}}}{11/2007 - 02/2009}{Embedded Sofware Developer}{Hanoi, Vietnam}
    \item Using PIC16F/33F serial, RF module, LED 7 Segment to develop some utility solution.
    \end{rSubsection}

  \end{rSection}
  \begin{rSection}{Other activity}
    {\bf Side project:} \\
    \textbf{\href{https://github.com/truongpt/video_watermarking}{- Video watermarking}}: Using motion vector to insert specific data, the implementation uses H.264 video format. \\
    \textbf{\href{https://github.com/truongpt/omxtranscoder}{- Video transcoder}}: Proof Of Concept transcode from MPEG2 to H.264 to reduce bitrate by using Raspberry PI . \\
    {\bf Blogger}: \href{https://gravieb.wordpress.com/}{https://gravieb.wordpress.com} \\
    {\bf Sefl study}: \href{https://www.coursera.org/account/accomplishments/verify/68DQB2KRJ5Y6}{Fundamentals of Digital Image and Video Processing}, \href{https://www.coursera.org/account/accomplishments/verify/S9U9FDD57TXM}{Machine Learning}, further study compiler \& RTOS. \\
    {\bf Book reading}: Technical, Physics, Politics, Soft skill. \\

  \end{rSection}

\end{document}
